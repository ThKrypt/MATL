\matl{!} & 1--2 (1 / 2) & 1 & If $1$ input: \matlab+.'+ (\matlab+transpose+), or \matlab+permute(..., [2 1 ...])+ for multidimensional arrrays. If $2$ inputs: \matlab+permute+. If the second input is 1, 2 or 3 it indicates which dimension is not permuted for a 3D array; that is, it corresponds to [1 3 2], [3 2 1] or [2 1 3] respectively \\
\matl{X!} & 1--2 (2) & 1 & \matlab+rot90+ \\
\matl{Y!} & 1 & 0--2 (2) & \matlab+system+ \\
\matl{Z!} & 1 & 1 & \matlab+full+ \\
\matl{X"} & 2-- (3) & 1 & \matlab+repmat+ \\
\matl{Y"} & 2-- (2 / 3) & 1 & \matlab+repelem+ (run-length decoding). With $2$ inputs: if the first or second inputs are empty the output is empty. Else, the second input is linearized if needed; and if the first or second input is shorter than the other, entries are repeated cyclically to match lengths \\
\matl{Z"} & 1 & 1 & \matlab+blanks+. It allows vector input, and in that case it produces a matrix or multidimensional array \\
\matl{X\#} & 0 & 0 & display stack as a cell array \\
\matl{Y\#} & 2-- (2 / 3) & 0 & (i) With $2$ inputs: \matlab+sound+. If second input is \matl{T}, it is interpreted as \matl{44100}. If second input is \matl{F}, it is interpreted as \matl{44100} and \matlab+audiowrite+ is called instead of \matlab+sound+, with file name \matlab+'audio.wav'+. (ii) With $3$ inputs: if third input is a truthy scalar, \matlab+soundsc+ is called with the first two inputs. If third input is a numeric vector of size $2$, \matlab+soundsc+ is called with the three inputs. In both cases, \matl{T} in second input is interpreted as \matl{44100}. If third input is a string, it defines a file name, and \matlab+audiowrite+ is called with that file name and the other two inputs. (iii) With more than $3$ inputs: \matlab+audiowrite+ is called using the third input as file name. Further inputs specify parameter-value pairs. \matl{T} in second input is interpreted as \matl{44100}.  \\
\matl{Z\#} & 1--3 (1 / 2) & 0 & Writes first input to file \comp{inout}, creating it if necessary. If the file exists, by default its previous contents are overwritten. If the input is a numeric or char array each entry is treated as a raw byte and written. Behaviour is undefined for entries exceeding $255$. If the input is a cell array, the contents of each cell are written, with a byte $10$ (representing newline) in between. With $2$ inputs: second input specifies file name; it is converted to char is necessary; if empty defaults to \comp{inout}. With $3$ inputs: third input specifies whether previous contents of the file should be kept \\
\matl{X\$} & 1 & 1 & (i) For non-char input: \matlab+sym+. (ii) For char input: \matlab+str2sym+. \sa \matl{Y\$} \\
\matl{Y\$} & 2 & 1 & (i) \matlab+char(vpa(str2sym(...), ...))+ for char input, or \matlab+char(vpa(...))+ for non-char input. Second input is number of significant digits. (ii) If the first input is a number or a string representing a real or imaginary scalar and the second input (number of significant digits) is negative, the latter is interpreted as its absolute value, and the output will give the actual digits without rounding (so inputs \matlab+'5/3'+ and \matlab+-4+ will produce \matlab+'1.666'+, not \matlab+'1.667'+). \sa \matl{X\$} \\
\matl{Z\$} & 0--1 (0 / 1) & 1 & Reads bytes from specifed file. Each individual byte is then converted to a char, and the output is a row vector of char. If $0$ inputs or empty input: file name is \comp{inout}. \\
\matl{X\%} & 1 & 1 & class of input (\matlab+class+ with one input) \\
\matl{Y\%} & 2--3 (2) & 1 & \matlab+cast+. This function allows strings in second input to be replaced by numbers, as follows: 1: \matlab+'uint8'+, 2: \matlab+'int8'+, 3: \matlab+'uint64'+, 4: \matlab+'int64'+, 5: \matlab+'uint16'+, 6: \matlab+'int16'+, 7: \matlab+'uint32'+, 8: \matlab+'int32'+, 9: \matlab+'double'+, 10: \matlab+'single'+ \\
\matl{Z\%} & 2 & 1 & \matlab+typecast+. This function allows strings in second input to be replaced by numbers, as follows: 1: \matlab+'uint8'+, 2: \matlab+'int8'+, 3: \matlab+'uint64'+, 4: \matlab+'int64'+, 5: \matlab+'uint16'+, 6: \matlab+'int16'+, 7: \matlab+'uint32'+, 8: \matlab+'int32'+, 9: \matlab+'double'+, 10: \matlab+'single'+ \\
\matl{X\&} & 2--4 (2 / 3) & 1--3 (1) & \matlab+intersect+. Uses the \matlab+'stable'+ flag by default. If one input is char and the other is numeric, the latter is converted to char. This function allows flag strings in third and subsequent inputs to be replaced by numbers, as follows: 1: \matlab+'rows'+, 2: \matlab+'stable'+, 3: \matlab+'sorted'+ \\
\matl{Y\&} & 0-- (2 / $^\ddagger$) & 1 & \matlab+&+ (\matlab+and+), element-wise with singleton expansion \\
\matl{Z\&} & 2--3 (2) & 1 & \matlab+bitand+ (bitwise 'and'), element-wise with singleton expansion. If first or second inputs are \matlab+char+ they are converted to \matlab+double+. \matlab+double+ inputs are rounded. If the inputs are \matlab+double+ they can have negative values, and in that case they must be in the range from \matlab+-2^52+ up to \matlab+2^52-1+. \sa \matl{Z|}, \matl{Z\textasciitilde{}} \\
\matl{X'} & 1-- (2 / 2) & 0-- (1) & execute Matlab function specified by last input, using the rest of the inputs as arguments. \\
\matl{Y'} & 1 & 2 (2 / 2nd) & run-length encoding (inverse of \matlab+repelem+). Input may be an array or cell array. Numeric values must be finite \\
\matl{Z'} & 0--1 (0 / 1) & 1 & \matlab+now+. With $1$ input: the input should be numeric with values from 1 to 6, which are used as indices into the output of \matlab+clock+ \\
\matl{(} & 3-- (3 / 4) & 1 & assignment \matlab+( )+ indexing. Null assignment (\matlab+x(...) = []+) can only be done with a single index. \sa \matl{Y(}, \matl{Z(} \\
\matl{X(} & 3-- (3) & 1 & assignment \matlab+{ }+ indexing \\
\matl{Y(} & 2-- (3) & 1 & assignment \matlab+(..., :)+ indexing. Null assignment (\matlab+x(..., :) = []+) can only be done with a single index (in addition to the implicit colon). \sa \matl{(}, \matl{Z(} \\
\matl{Z(} & 2-- (3) & 1 & assignment \matlab+(:, ...)+ indexing. Null assignment (\matlab+x(:, ...) = []+) can only be done with a single index (in addition to the implicit colon). \sa \matl{(}, \matl{Y(} \\
\matl{)} & 2-- (2) & 1--2 (1 / 2) & reference \matlab+( )+ indexing. If $2$ outputs: only one input index can be used. The second output produces the "complementary" array \matlab+y=x; y(ind)=[]+, where \matlab+x+ and \matlab+ind+ are the inputs. \sa \matl{Y)}, \matl{Z)} \\
\matl{X)} & 2-- (2) & 0-- ($^\bigtriangledown$) & reference \matlab+{ }+ indexing \\
\matl{Y)} & 1-- (2) & 1--2 (1 / 2) & reference \matlab+(..., :)+ indexing. If $2$ outputs: only one input index can be used. The second output produces the "complementary" array \matlab+y=x; y(ind,:)=[]+, where \matlab+x+ and \matlab+ind+ are the inputs. \sa \matl{)}, \matl{Z)} \\
\matl{Z)} & 1-- (2) & 1--2 (1 / 2) & reference \matlab+(:, ...)+ indexing. If $2$ outputs: only one input index can be used. The second output produces the "complementary" array \matlab+y=x; y(:,ind)=[]+, where \matlab+x+ and \matlab+ind+ are the inputs. \sa \matl{)}, \matl{Y)} \\
\matl{*} & 1-- (2 / 1) & 1 & \matlab+.*+ (\matlab+times+), element-wise with singleton expansion. If $1$ input: a second input is used given by the first transposed \\
\matl{X*} & 2 & 1 & \matlab+kron+ \\
\matl{Y*} & 2 & 1 & matrix product, \matlab+*+ (\matlab+mtimes+) \\
\matl{Z*} & 1-- (2 / $^\ddagger$) & 1 & Cartesian product. Given a number $n$ of arrays of possibly different sizes, generates an $n$-column matrix whose rows describe all combinations of elements taken from those arrays \\
\matl{+} & 1-- (2 / 1) & 1 & \matlab|+| (\matlab+plus+), element-wise with singleton expansion. If $1$ input: a second input is used given by the first transposed \\
\matl{Y+} & 2--3 (2 / 3) & 1 & \matlab+conv2+. Doesn't allow two-vector and one matrix mode. Converts first two inputs to \matlab+double+. This function allows flag strings in third and subsequent inputs to be replaced by numbers, as follows: 1: \matlab+'same'+, 2: \matlab+'valid'+, 3: \matlab+'full'+. \sa \matl{Z+} \\
\matl{Z+} & 2--3 (2 / 3) & 1 & (i) If $2$ inputs: \matlab+conv2(..., 'same')+. (ii) If $3$ inputs: \matlab+cconv+. Non-vector inputs are linearized into column vectors. The output is of type \matlab+double+. Integer inputs are guaranteed to give exact integer results, up to \matlab+double+ data type limitations. (i,ii) Inputs are converted to \matlab+double+. \sa \matl{Y+} \\
\matl{X,} & 1 & 1 & \matlab+cos+ \\
\matl{Y,} & 1 & 1 & \matlab+sin+ \\
\matl{Z,} & 1 & 1 & \matlab+tan+ \\
\matl{-} & 1--2 (2 / 1) & 1 & \matlab+-+ (\matlab+minus+), element-wise with singleton expansion. If $1$ input: a second input is used given by the first transposed \\
\matl{X-} & 2--4 (2 / 3) & 1--2 (1) & \matlab+setdiff+. Uses the \matlab+'stable'+ flag by default. If one input is char and the other is numeric, the latter is converted to char. This function allows flag strings in third and subsequent inputs to be replaced by numbers, as follows: 1: \matlab+'rows'+, 2: \matlab+'stable'+, 3: \matlab+'sorted'+. \sa \matl{X\textasciitilde{}} \\
\matl{Y-} & 2 & 1--2 (1 / 2) & \matlab+deconv+ \\
\matl{Y.} & 0--1 (1) & 0 & \matlab+pause+ (without outputs) \\
\matl{Z.} & 2--3 (2) & 1 & \matlab+bitget+. If first input is \matlab+char+ it is automatically converted to \matlab+double+ \\
\matl{/} & 2 & 1 & \matlab+./+ (\matlab+rdivide+), element-wise with singleton expansion \\
\matl{X/} & 1 & 1 & \matlab+angle+ \\
\matl{Y/} & 2 & 1 & right matrix division, \matlab+/+ (\matlab+mrdivide+) \\
\matl{Z/} & 1--3 (1) & 1 & \matlab+unwrap+ \\
\matl{X0} & 1 & 1 & predefined literal depending on input \\
\matl{Y0} & 1 & 1 & predefined literal depending on input \\
\matl{X1} & 1 & 1 & predefined literal depending on input \\
\matl{Y1} & 1 & 1 & predefined literal depending on input \\
\matl{X2} & 1 & 1 & predefined literal depending on input \\
\matl{Y2} & 1 & 1 & predefined literal depending on input \\
\matl{X3} & 1 & 1 & predefined literal depending on input \\
\matl{Y3} & 1 & 1 & predefined literal depending on input \\
\matl{X4} & 1 & 1 & predefined literal depending on input \\
\matl{Y4} & 1 & 1 & predefined literal depending on input \\
\matl{X5} & 1 & 1 & predefined literal depending on input \\
\matl{Y5} & 1 & 1 & predefined literal depending on input \\
\matl{X6} & 1 & 1 & predefined literal depending on input \\
\matl{Y6} & 1 & 1 & predefined literal depending on input \\
\matl{X7} & 1 & 1 & predefined literal depending on input \\
\matl{X8} & 1 & 1 & predefined literal depending on input \\
\matl{X9} & 1 & 1 & predefined literal depending on input \\
\matl{:} & 1--3 (1 / 2) & 1 & \matlab+colon+ (with three inputs \matlab+x+, \matlab+y+, \matlab+z+ produces \matlab+x:y:z+; with two inputs \matlab+x+, \matlab+y+ produces \matlab+x:y+). If one input: produces \matlab+1:x+, or \matlab+' ':x+ if \matlab+x+ is char. For a single cell-array input, the contents of the cell array are interpreted as the actual inputs. \\
\matl{X:} & 1 & 1 & linearize to column array (index with \matlab+(:)+) \\
\matl{Y:} & 1 & 0-- ($^\triangle$) & generate comma-separated list from cell array (index with \matlab+{:}+) and push each element onto stack \\
\matl{Z:} & 2--4 (3) & 1 & \matlab+bitset+. If first input is \matlab+char+ it is automatically converted to \matlab+double+ \\
\matl{X;} & 1 & 1 & \matlab+acos+ \\
\matl{Y;} & 1 & 1 & \matlab+asin+ \\
\matl{Z;} & 2 & 1 & \matlab+atan2+, element-wise with singleton expansion \\
\matl{<} & 1--2 (2 / 1) & 1 & \matlab+<+ (\matlab+lt+), element-wise with singleton expansion. For complex values, the real parts are compared. If $1$ input: a second input is used given by the first transposed \\
\matl{X<} & 1--3 (1) & 1--2 (1 / 2nd) & \matlab+min+. If $2$ inputs: element-wise with singleton expansion. With more than $2$ inputs, the second is replaced by \matl{[]}. This function does not support flags \matlab+'omitnan'+ or \matlab+'includenan'+. Output is \matlab+char+ if there is $1$ input and it is \matlab+char+, or if there are $2$ inputs and both are \matlab+char+, or if there are $3$ inputs, second is empty and third is \matlab+char+. \sa \matl{X>}, \matl{Xl} \\
\matl{Y<} & 1--3 (1 / 2) & 1 & \matlab+cummin+. Output is \matlab+char+ if first input is. \sa \matl{Y>} \\
\matl{=} & 1--2 (2 / 1) & 1 & \matlab+==+ (\matlab+eq+), element-wise with singleton expansion. If $1$ input: a second input is used given by the first transposed \\
\matl{X=} & 0-- (2 / $^\ddagger$) & 1 & \matlab+isequal+. Works for any number of inputs \\
\matl{Y=} & 2 & 1 & \matlab+strcmp+. If first or second inputs are numeric they are converted to char \\
\matl{>} & 1--2 (2 / 1) & 1 & \matlab+>+ (\matlab+gt+), element-wise with singleton expansion. For complex values, the real parts are compared. If $1$ input: a second input is used given by the first transposed \\
\matl{X>} & 1--3 (1) & 1--2 (1 / 2nd) & \matlab+max+. If $2$ inputs: element-wise with singleton expansion. With more than $2$ inputs, the second is replaced by \matl{[]}. This function does not support flags \matlab+'omitnan'+ or \matlab+'includenan'+. Output is \matlab+char+ if there is $1$ input and it is \matlab+char+, or if there are $2$ inputs and both are \matlab+char+, or if there are $3$ inputs, second is empty and third is \matlab+char+. \sa \matl{X<}, \matl{Xl} \\
\matl{Y>} & 1--3 (1 / 2) & 1 & \matlab+cummax+. Output is \matlab+char+ if first input is. \sa \matl{Y<} \\
\matl{Y?} & 0 & 1 & answer why. Sort of \\
\matl{Z?} & 1--6 (3) & 1 & \matlab+sparse+. If $3$ inputs and third input is \matlab+char+, the output is converted to \matlab+char+ \\
\matl{Y@} & 1--2 (1 / 2) & 1 & If $1$ input: \matlab+perms+. If $2$ inputs: variations (without repetition). In either case, the results are sorted \\
\matl{Z@} & 1--3 (1 / 2) & 1 & \matlab+randperm+ (produces a row vector as output). If $3$ inputs: third input indicates number of permutations, each on a different row. If first input is char it is interpreted as population (not as size) \\
\matl{A} & 1--2 (1 / 2) & 1 & \matlab+all+. \sa \matl{XA} \\
\matl{XA} & 1 & 1 & \matlab+all(..., 1)+. \sa \matl{A} \\
\matl{YA} & 2--3 (2 / 3) & 1 & (i) \matlab+dec2base+. (ii) If second input has more than one element: it defines the symbols, which can be characters or numbers. The number of symbols defines the base, which can exceed $36$. (iii) If second input is a negative number \matlab+-n+: it is interpreted as symbols \matlab+0:n-1+ (case ii). (i, ii) Base \matl{0} is interpreted as \matl{10}, \matl{1} as \matl{16}, \matl{F} as \matl{0:9}, \matl{T} as \matl{0:15}. \sa \matl{ZA}, \matl{Za} \\
\matl{ZA} & 2 & 1 & (i) \matlab+base2dec+. (ii) If second input has more than one element: it defines the symbols, which can be characters (case-sensitive) or numbers. The number of symbols defines the base, which can exceed $36$. (iii) If second input is a negative number \matlab+-n+: it is interpreted as symbols \matlab+0:n-1+ (case ii). (i, ii, iii) Non-recognized digits are ignored. Base \matl{0} is interpreted as \matl{10}, \matl{1} as \matl{16}, \matl{F} as \matl{0:9}, \matl{T} as \matl{0:15}. \sa \matl{YA}, \matl{Za} \\
\matl{B} & 1--2 (1 / 2) & 1 & \matlab|logical(dec2bin(...)-'0')|. \sa \matl{YB} \\
\matl{XB} & 1 & 1 & \matlab|bin2dec(char(logical(...)+'0'))|. Works also for cell array input. \sa \matl{ZB} \\
\matl{YB} & 1--2 (1 / 2) & 1 & \matlab|dec2bin|. \sa \matl{B} \\
\matl{ZB} & 1 & 1 & \matlab|bin2dec|. \sa \matl{XB} \\
\matl{YC} & 2--4 (2) & 1 & \matlab+im2col+. If the second input is a scalar \matlab+n+, it is transformed into \matlab+[1 n]+ if the first input is a row vector, or to \matlab+[n 1]+ otherwise. First input can also be a cell array. \sa \matl{ZC} \\
\matl{ZC} & 2--3 (2) & 1 & \matlab+im2col(..., 'distinct')+. If the second input is a scalar n, it is transformed into [1 n] if the first input is a row vector, or to [n 1] otherwise. First input can also be a cell array. \sa \matl{YC} \\
\matl{D} & 0-- (1) & 0--1 (0 / 1) & (i) With $0$ outputs: If $1$ input: \matlab+disp(num2str(..., '%.15g '))+. If several inputs: \matlab+disp(num2str(eachInput,lastInput))+, where \matlab+eachInput+ loops over all inputs but the last. In either case, (nested) cell arrays are (recursively) unboxed in linear order. Most characters below 32 are replaced by space. Symbolic data is displayed directly. (ii) With $1$ output: \matlab+mat2str+. Empty arrays are always shown as \matlab+[]+, \matlab+''+ or \matlab+{}+. Most input characters below 32 are replaced by space. Cell arrays are converted to string representation too. Optional second, third and fourth inputs respectively specify column separator, row separator and whether the separators should be used for non-cell arrays too. Second and third output are converted to char is needed. \sa \matl{XD}, \matl{YD}, \matl{ZD} \\
\matl{XD} & 0-- ($^\ddagger$ / 1) & 0 & \matlab+disp(num2str(eachInput, '%.15g '))+, where \matlab+eachInput+ loops over all inputs. (Nested) cell arrays are (recursively) unboxed in linear order. Most characters below 32 are replaced by space. Symbolic data is displayed directly. \sa \matl{D}, \matl{YD}, \matl{ZD} \\
\matl{YD} & 1-- (2 / 1) & 0--2 (1 / 0) & \matlab+sprintf+ with format string as last input. If $0$ outputs: prints to screen using \matlab+fprintf(...)+. \sa \matl{D}, \matl{XD}, \matl{ZD} \\
\matl{ZD} & 0-- (1) & 0 & \matlab+disp+ for each input. For char input, most characters below 32 are replaced by space. \sa \matl{D}, \matl{XD}, \matl{YD} \\
\matl{E} & 1 & 1 & \matlab|(...)*2| \\
\matl{XE} & 3 & 1 & With numeric or char inputs, replace in first input all occurrences of each element of the second input by the corresponding element of the third input. The third input may be longer than the second, and then the extra elements are ignored. Or it may have a single element, and then it is implicitly replicated. Output has the same class and size as the first input. If the three inputs are cell arrays of strings (the second may also be a string instead of a cell array of strings), each string of the first input is considered atomic, that is, replacing is based on whole strings. If the first input is a cell array and the others are numeric or char, replacing is done on each cell's contents as if the cell's contents were the first input \\
\matl{YF} & 1 & 1--2 (1 / 2) & (i) With $1$ output: exponents of prime factor decomposition, without skipping primes. If the input is a numeric array it is linearized, and each result is in an output row. If any entry in the input is negative, the absolute value is taken, and the output contains only non-zero exponents. (ii) With $2$ outputs: first output gives the prime factors, second gives the exponents. Primes that are not factors are skipped, as are their zero exponents. If any entry in the input is negative, the absolute value is taken, and all intermediate primes are included, possibly with zero exponents. \sa \matl{Yf} \\
\matl{ZF} & 1--3 (1 / 2) & 1 & (i) With $1$ or $2$ inputs: \matlab+nfft+. If first input is a vector and second input is a scalar, a $1$ is prepended or appended to the second input according to the vector orientation (so the result is the same as that of \matlab+fft+). (ii) With $3$ inputs: \matlab+fft+ \\
\matl{G} & 0--1 ($^\sqcup$ / 0) & 0-- ($^\sqcap$) & paste from user-input clipboard G. If $0$ input arguments: addresses all levels. If $1$ input argument: addresses specified level. In either of those cases, if clipboard G has no levels one user-input is implicitly taken to fill the first level \\
\matl{XG} & 1-- (1 / 2) & 0 & \matlab+plot+. Calls \matlab+drawnow+ to update figure immediately. With one input, or with several inputs the second of which is a string: if the first input is complex (even with zero imaginary part), \matlab+axis equal+ is also called. \\
\matl{YG} & 2-- (2 / 3) & 0 & \matlab+imwrite+, \matlab+imagesc+, \matlab+image+ or \matlab+imshow+. (i) If last input is a scalar: \matlab+0+ corresponds to \matlab+imwrite+, \matlab+1+ to \matlab+imagesc+, \matlab+2+ to \matlab+image+ and \matlab+3+ to \matlab+imshow+. The corresponding function is called with the remaining inputs. (ii) If last input is numeric or logical and not a scalar: \matlab+imshow+ is called with all inputs. (iii) If last input is char: \matlab+imwrite+ is called with all inputs. (i, iii) For \matlab+imwrite+, the first input of type char is interpreted as file name. If it has no extension '.png' is added; if it's empty it is replaced by 'image.png'; and if non existent 'image.png' is used as final input. (i, ii, iii) For \matl{imshow} and \matl{imwrite}, if the second input is logical it is converted to \matlab+double+. If it is numeric, has the shape of a colormap, and has some entry greater than $1$, it is normalized by converting to \matlab+uint8+, then to \matlab+double+, and then dividing by $255$. For \matlab+imagesc+ and \matlab+image+, the function call is followed by \matlab+axis ij, axis image+. For \matlab+imagesc+, \matlab+image+ and \matlab+imshow+, \matlab+drawnow+ is called to update figure immediately \\
\matl{ZG} & 1-- (2 / 3) & 0--1 (0) & Depending on numeric last input, calls a graphic function or \matlab+format+ with the remaining inputs.  $0$: \matlab+format+.  $1$: \matlab+axis+. Calls \matlab+drawnow+ to update figure immediately. Flag strings in first to second-last inputs can be replaced by numbers, as follows: 1: \matlab+'equal'+, 2: \matlab+'image'+, 3: \matlab+'square'+, 4: \matlab+'ij'+, 5: \matlab+'xy'+, 6: \matlab+'normal'+, 7: \matlab+'off'+, 8: \matlab+'on'+, 9: \matlab+'tight'+, 10: \matlab+'manual'+, 11: \matlab+'fill'+, 12: \matlab+'auto'+, 13: \matlab+'vis3d'+.  $2$: \matlab+colormap+. If the first input is numeric, has the shape of a colormap, and has some entry greater than $1$, it is normalized by converting to \matlab+uint8+, then to \matlab+double+, and then dividing by $255$. With $3$ inputs (including the last input 2), the second input specifies colormap size. With $0$ outputs, calls \matlab+drawnow+ to update figure immediately.  $3$: \matlab+hold+. Flag strings in first input can be replaced by numbers, as follows: 1: \matlab+'on'+, 2: \matlab+'off'+.  $4$: \matlab+grid+. Flag strings in first input can be replaced by numbers, as follows: 1: \matlab+'on'+, 2: \matlab+'off'+.  $5$: \matlab+colorbar+. If the first input is logical (not necessarily scalar) it activates or deactivates the colorbar \\
\matl{H} & 0 & 0-- ($^\dagger$) & paste from clipboard H \\
\matl{XH} & 0-- (1 / 2) & 0 & copy to clipboard H \\
\matl{YH} & 1-- (2 / 4) & 0 & Depending on numeric last input, calls a plotting function with the remaining inputs.  $0$: \matlab+plot3+.  $1$: \matlab+surf+.  $2$: \matlab+mesh+.  $3$: \matlab+stem+.  $4$: \matlab+stairs+.  $5$: \matlab+bar+.  $6$: \matlab+loglog+.  After that, \matlab+drawnow+ is called to update figure immediately \\
\matl{I} & 0 & 0-- ($^\dagger$) & paste from clipboard I \\
\matl{XI} & 0-- (1) & 0 & copy to clipboard I \\
\matl{YI} & 3--4 (3 / 4) & 1 & \matlab+col2im+. Uses \matlab+'distinct'+ option by default. Second and third inputs may be scalars, and then they are interpreted as numbers of columns. Third input may be a two-vector with product less then the number of elements of first input, and then it is appropriately scaled. This function allows flag strings in fourth input to be replaced by numbers, as follows: 1: \matlab+'distinct'+, 2: \matlab+'sliding'+ \\
\matl{ZI} & 1-- (2 / 3) & 1-- (1) & Depending on numeric last input, calls an image processing function with the remaining inputs.  $0$: \matlab+imfill+. If first input is logical or numerical it is converted to char.  $1$: \matlab+bwlabeln+.  $2$: \matlab+imdilate+. This function allows second input to be number $4$, $5$, $8$ or $9$, which is interpreted as the corresponding neighbourhood mask.  $3$: \matlab+imerode+. This function allows second input to be number $4$, $5$, $8$ or $9$, which is interpreted as the corresponding neighbourhood mask.  $4$: \matlab+bweuler+.  \\
\matl{J} & 0 & 0-- ($^\dagger$) & paste from clipboard J \\
\matl{XJ} & 0-- (1) & 0 & copy to clipboard J \\
\matl{K} & 0 & 0-- ($^\dagger$) & paste from clipboard K \\
\matl{XK} & 0-- (1) & 0 & copy to clipboard K \\
\matl{L} & 1 & 0-- ($^\dagger$) & paste from multi-level clipboard L. Input specifies level \\
\matl{XL} & 1-- (2 / 3) & 0 & copy to multi-level clipboard L. Last input specifies level \\
\matl{YL} & 1-- (2 / 3) & 1-- (1) & \matlab+gallery+ with matrix name as last input. Also includes \matlab+magic+, \matlab+hilb+, \matlab+invhilb+, \matlab+hadamard+, \matlab+pascal+, \matlab+spiral+. This function allows some strings in last input to be replaced by numbers, as follows:  1: \matlab+'spiral'+, 2: \matlab+'pascal'+, 3: \matlab+'magic'+, 4: \matlab+'hadamard'+, 5: \matlab+'circul'+, 6: \matlab+'gcdmat'+, 7: \matlab+'minij'+, 8: \matlab+'hilb'+, 9: \matlab+'invhilb'+, 10: \matlab+'tridiag'+, 11: \matlab+'ris'+ \\
\matl{M} & 1 & 0-- ($^\ast$) & paste from function-input clipboard M. Input specifies level ($1$ to $4$) or individual input ($5$ or larger) \\
\matl{XM} & 1--2 (1) & 1--3 (1) & \matlab+mode+. First input can be a cell array of strings \\
\matl{N} & 0 & 1 & number of elements in the stack \\
\matl{XN} & 2 & 1 & \matlab+nchoosek+. This interprets first input as an array (even if it is a single number). For inputs \matlab+x+ and \matlab+k+, if \matlab+x+ has less than \matlab+k+ elements or if \matlab+k+ is non-positive the result is an empty array. \sa \matl{Xn} \\
\matl{YN} & 0-- (0) & 1 & \matlab+NaN+ function. If $0$ inputs: produces literal \matlab+NaN+. \\
\matl{ZN} & 1 & 1 & \matlab+isnan+ \\
\matl{O} & 0-- (0 / 2) & 1 & \matlab+zeros+ (if $0$ inputs: produces output $0$) \\
\matl{XO} & 1--4 (2) & 1 & \matlab+datestr+. \sa \matl{YO}, \matl{ZO} \\
\matl{YO} & 1--6 (1 / 2) & 1 & \matlab+datenum+. With $2$ inputs, if the second input is numeric it is interpreted as a format specifier as in \matlab+datestr+. \sa \matl{XO}, \matl{ZO} \\
\matl{ZO} & 1--3 (1 / 2) & 1--6 (1) & \matlab+datevec+. \sa \matl{XO}, \matl{YO} \\
\matl{P} & 1--2 (1 / 2) & 1 & \matlab+flip+. \sa \matl{XP} \\
\matl{XP} & 1 & 1 & \matlab+flipud+. \sa \matl{P} \\
\matl{YP} & 0 & 1 & \matlab+pi+ \\
\matl{ZP} & 1--5 (2 / 3) & 1--2 (1) & (i) \matlab+pdist2+. Only predefined distance functions are allowed. This function allows flag strings in the third input to be replaced by numbers, as follows:  1: \matlab+'cityblock'+, 2: \matlab+'hamming'+, 3: \matlab+'chebychev'+, 4: \matlab+'correlation'+, 5: \matlab+'cosine'+, 6: \matlab+'seuclidean'+, 7: \matlab+'minkowski'+, 8: \matlab+'mahalanobis'+, 9: \matlab+'spearman'+, 10: \matlab+'jaccard'+, 11: \matlab+'euclidean'+. (ii) With $1$ input and $1$ output: row vector containing all entries below the main diagonal of a matrix \\
\matl{Q} & 1 & 1 & \matlab|(...)+1| \\
\matl{XQ} & 2--6 (3) & 1 & \matlab+accumarray+. If first input is \matlab+char+ it is converted to \matlab+double+. The third input may be omitted, and then it is interpreted as \matlab+[]+. Fourth/third argument specifies an anonymous function, as follows:  1: \matlab+'@sum'+, 2: \matlab+'@mean'+, 3: \matlab+'@(x){sort(x).'}'+, 4: \matlab+'@max'+, 5: \matlab+'@min'+, 6: \matlab+'@prod'+, 7: \matlab+'@(x){x.'}'+, 8: \matlab+'@(x){x}'+, 9: \matlab+'@(x){sort(x)}'+, 10: \matlab+'@(x)x(1)'+, 11: \matlab+'@(x)x(end)'+, 12: \matlab+'@(x){cumsum(x).'}'+, 13: \matlab+'@(x){cumprod(x)}'+, 14: \matlab+'@nansum'+, 15: \matlab+'@nanmean'+, 16: \matlab+'@nanmax'+, 17: \matlab+'@nanmin'+, 18: \matlab+'@(x){cummax(x).'}'+, 19: \matlab+'@(x){cummin(x)}'+, 20: \matlab+'@(x){cummax(x).'}'+, 21: \matlab+'@(x){cummin(x)}'+ \\
\matl{YQ} & 1--2 (2) & 1--2 (2) & \matlab+rat+ \\
\matl{ZQ} & 1--4 (2 / 1) & 1--2 (1) & (i) If $1$ input: \matlab+roots+. (ii) If $2$ inputs: \matlab+polyval+. (iii) If $3$ inputs: \matlab+polyfit+. (iv) If $4$ inputs: \matlab+inpolygon+ \\
\matl{R} & 1--2 (1 / 2) & 1 & \matlab+triu+. \sa \matl{XR}. \\
\matl{XR} & 1--2 (1 / 2) & 1 & \matlab+triu(..., 1)+. If $2$ inputs: builds an upper triangular or symmetric matrix from a vector (or from an array in linear order). Second input indicates if the diagonal is filled/not and if the matrix is made symmetric, as follows. 0: don't use diagonal, don't make symmetric. 1: use diagonal, don't make symmetric. 2: don't use diagonal, make symmetric. 3: use diagonal, make symmetric. \sa \matl{R}, \matl{ZR} \\
\matl{YR} & 1--2 (1 / 2) & 1 & \matlab+tril+. \sa \matl{ZR}. \\
\matl{ZR} & 1--2 (1 / 2) & 1 & \matlab+tril(..., -1)+. If $2$ inputs: builds a lower triangular or symmetric matrix from a vector (or from an array in linear order). Second input indicates if the diagonal is filled/not and if the matrix is made symmetric, as follows. 0: don't use diagonal, don't make symmetric. 1: use diagonal, don't make symmetric. 2: don't use diagonal, make symmetric. 3: use diagonal, make symmetric. \sa \matl{XR}, \matl{YR}. \\
\matl{S} & 1--3 (1) & 1--2 (1 / 2nd) & sort an array (\matlab+sort+) / sort an array based on another. (i) Single-array mode works like Matlab's \matlab+sort+. If $2$ inputs, a negative value of the second input corresponds to descending order. If first input is a cell array and the first cell contains a char array, the rest of the cells' contents are converted to char. (ii) If the first input is a cell array and the first cell contains a numeric array, single-array numeric mode is used. The first input is linearized if it's not a vector, and its contents are linearized for the purposes of sorting. The first input is then sorted in lexicographic order, ignoring other inputs. (iii) In two-array mode, this function takes as first $2$ inputs an array and a vector array which is not char, both with the same number of elements. If the first array is not a vector it is linearized. The second vector is sorted and its order is applied to the first. An optional third input specifies direction as a string, or as a negative number in the non-singleton dimension of the second vector. The outputs are the two sorted arrays. (In two-array mode, if the two input arrays are scalar and the second can be interpreted as a dimension there is no inconsistency with single array mode, as the result is the same). \sa \matl{XS} \\
\matl{XS} & 1--2 (1) & 1--2 (1 / 2nd) & \matlab+sortrows+. \sa \matl{S} \\
\matl{YS} & 2--3 (2 / 3) & 1 & \matlab+circshift+. If second input is a scalar and there's no third input, the shift is applied along the first non-singleton dimension. This function also allows first input a 2D array; third input a scalar specifying dimension; and second input a vector or array specifying the shift for each position in the other dimension \\
\matl{ZS} & 1--3 (1 / 3) & 1 & (i) With $1$ input: \matlab+sign+. (ii) With $2$ inputs: \matlab+fftshift+. Second input equal to $0$ corresponds to \matlab+fftshift+ with $1$ input. (iii) With $3$ inputs: \matlab+linspace+ \\
\matl{YT} & 1--2 (1 / 2) & 1 & \matlab+toeplitz+. Output is \matlab+char+ if any input is \\
\matl{U} & 1 & 1--2 (1) & (i) For char input: \matlab+str2num+ with content checking. Most characters below 32 are replaced by space (as in \matl{D}). The input content is then checked. If it fails, \matlab+[]+ is returned. Else \matlab+str2num+ is applied. If that fails, the input string is evaluated. If that also fails, \matlab+[]+ is returned. The second output is supported in all cases. (ii) For numeric input: \matlab+(...).^2+ \\
\matl{XU} & 1 & 1 & \matlab+str2double+ \\
\matl{V} & 1--2 (1 / 2) & 1 & \matlab+num2str+. Uses format \matlab+'%.15g '+ by default. To get Matlab's default format use \matlab+[]+ as format specification \\
\matl{W} & 1 & 1 & \matlab+2.^(...)+. Array power (or rather exponentiation) with base $2$ \\
\matl{XX} & 2--9 (2 / 3) & 1--6 ($^\Diamond$) & \matlab+regexp+. With $2$ inputs and $1$ output: \matlab+regexp(..., ..., 'match')+. If first or second inputs are numeric they are converted to char. If they are numeric or char arrays they are linearized into a row. If they are cells their contents are converted to char. \matlab+'names'+ output is not supported. This function allows flag strings in third and subsequent inputs to be replaced by numbers, as follows: 1: \matlab+'start'+, 2: \matlab+'split'+, 3: \matlab+'end'+, 4: \matlab+'once'+, 5: \matlab+'tokenExtents'+, 6: \matlab+'tokens'+, 7: \matlab+'match'+ \\
\matl{YX} & 3--5 (3 / 4) & 1 & \matlab+regexprep+. If first, second or third inputs are numeric they are converted to char. If they are numeric or char arrays they are linearized into a row. If they are cells their contents are converted to char \\
\matl{YY} & 0-- (0) & 1 & \matlab+inf+ function. If $0$ inputs: produces literal \matlab+inf+. \\
\matl{ZY} & 1 & 1 & \matlab+isinf+ \\
\matl{X[} & 2 & 1-- ($^\triangle$) & \matlab+ind2sub+ without input restrictions \\
\matl{\textbackslash } & 2 & 1--2 (1 / 2) & \matlab+mod+, element-wise with singleton expansion. With $2$ outputs: second output is \matlab+floor(.../...)+. For $1$ or $2$ outputs, if the first input is complex the real and imaginary parts are dealt with separately. \sa \matl{X\textbackslash }, \matl{o} \\
\matl{X\textbackslash } & 2 & 1 & \matlab|mod(...-1, ...)+1|, element-wise with singleton expansion. \sa \matl{\textbackslash } \\
\matl{Y\textbackslash } & 2 & 1 & left matrix division, \matlab+\+ (\matlab+mldivide+) \\
\matl{Z\textbackslash } & 1 & 1 & divisors of a number. For negative input the absolute value is used \\
\matl{X]} & 3-- (3) & 1 & \matlab+sub2ind+ without input restrictions \\
\matl{\textasciicircum{}} & 2 & 1 & \matlab+.^+ (\matlab+power+), element-wise with singleton expansion \\
\matl{X\textasciicircum{}} & 1 & 1 & \matlab+sqrt+ \\
\matl{Y\textasciicircum{}} & 2 & 1 & \matlab+^+ (\matlab+mpower+). If the second input is an array, the output is the sum of matrix powers with those exponents \\
\matl{Z\textasciicircum{}} & 2 & 1 & (i) Given an array and a number $n$, computes the Cartesian power of the array times itself $n$ times. (ii) If the first input is a number and the second input is an array of at least two elements, the inputs are interpreted in reverse order. When using this mode, caution is needed in case the second input (array) may become a scalar, because then mode (i) will be used. \sa \matl{Z*} \\
\matl{\_} & 1 & 1 & (i) If input is not \matlab+uint8+: unary \matlab+-+ (\matlab+uminus+), that is, output is the negative of the input. (ii) If input is \matlab+uint8+: converts to \matlab+double+ and divides by $255$ \\
\matl{Y`} & 0 & 0--1 (0) & \matlab+tic+. \sa \matl{Z`} \\
\matl{Z`} & 0--1 (0) & 0--1 (1) & \matlab+toc+. \sa \matl{Y`} \\
\matl{a} & 1--2 (1 / 2) & 1 & \matlab+any+. \sa \matl{Xa} \\
\matl{Xa} & 1 & 1 & \matlab+any(..., 1)+. \sa \matl{a} \\
\matl{Ya} & 2--4 (2 / 3) & 1 & (i) \matlab+padarray+. It allows the first input to be \matlab+char+; and then the output is also \matlab+char+. If the second input is \matlab+logical+ or the pad value is \matlab+char+ they are converted to \matlab+double+. This function allows flag strings in fourth input to be replaced by numbers, as follows: 1: \matlab+'pre'+, 2: \matlab+'post'+, 3: \matlab+'both'+. (ii) If the second input contains at least one negative or complex value, the array in the first input is unpadded. The array can only have two dimensions. The second input can have one or two entries, specifying dimensions. A negative value indicates unpad along that dimension. A complex value indicates unpad along the two dimensions. The third input specifies which values are considered padding; by default $0$ \\
\matl{Za} & 3--4 (3 / 4) & 1 & Converts the number represented by input 1 from the base specified by input 2 to that of input 3. Each base can be a number or a vector. In the first case the alphabet is from 0 to that number minus 1. If the second or third input equals \matl{T} or \matl{F}, it is respectively interpreted as \matlab+' ':'~'+ (all printable ASCII chars) or \matlab+[' ':'&' '(':'~']+ (all printable ASCII chars except single quote). Non-valid digits in the first input are discarded. An optional fourth input indicates number of digits of the result. First input can be a matrix or a cell array; and then the result is a matrix in which each row corresponds to a row of the input matrix, or to a cell of the input cell array in linear order. \sa \matl{YA}, \matl{ZA} \\
\matl{b} & 0-- (3 / 4) & 0 & bubble up element in stack. \sa \matl{w} \\
\matl{Yb} & 1-- (1 / 2) & 1--2 (1) & (i) \matlab+strsplit+. If second input is numeric it is converted to char. This function allows flag strings in third and successive inputs to be replaced by numbers, as follows: 1: \matlab+'CollapseDelimiters'+, 2: \matlab+'DelimiterType'+, 3: \matlab+'RegularExpression'+, 4: \matlab+'Simple'+. (ii) First input can be a cell array of strings, and then the result is a cell array of cell arrays of strings. Only one output is supported in this mode. (iii) First input can be numeric or logical. In this mode only one output is supported, the default delimiter is 0 or \matlab+false+, and the parameter \matlab+'CollapseDelimiters'+ can be used. The second input (delimiter) can be a single number or a numeric array, and in the latter case each element is a possible delimiter. If the first input is a row array the output will contain row arrays; otherwise it will contain column arrays \\
\matl{c} & 1-- (1) & 1 & \matlab+char+. Sparse input is first converted to full, and logical input is first converted to double. For cell array input, non-char cell contents are first converted to char. In all cases, non-integer values are rounded down \\
\matl{Xc} & 3-- (3 / $^\ddagger$) & 1 & \matlab+cat+. The dimension is the last input \\
\matl{Yc} & 1-- (2 / $^\ddagger$) & 1 & \matlab+strcat+. If not all inputs are numerical or logical, numerical or logical inputs are converted to char. If all inputs are numerical or logical the result is numerical or logical (with singleton expansion along the first dimension), and no trailing values are removed  \\
\matl{Zc} & 1--2 (1 / 2) & 1 & If cell input: \matlab+strjoin+. This function also allows input with numeric content. The first cell of the first input determines char or numeric mode. If that cell contains a char (resp. numeric) array, numeric (resp. char) contents in other cells, as well as the second input or its contents, are converted to char (resp. double). (i) Char mode corresponds to \matlab+strjoin+. (ii) Numeric mode is similar: it has a default delimiter, which is 0; or a single delimiter may be specified, which may be a scalar or an array; or a cell array of delimiters may be used. (i, ii) Both in char and in numeric mode, non-vector arrays are linearized, and the result is a row vector. Surplus delimiters are ignored. (iii) With a single, non-cell input: nonzeros are replaced by 35 (code point of \matlab+'#'+) and the result is converted to char \\
\matl{d} & 1--3 (1 / 2) & 1 & \matlab+diff+ with second and third input interchanged: second specifies dimension, third specifies difference order. \\
\matl{Xd} & 1--4 (1 / 2) & 1--2 (1) & If $1$ input and $1$ output: \matlab+diag+. Otherwise: \matlab+spdiags+, with char inputs automatically converted to \matlab+double+. With $2$ inputs and $1$ output, if the second input is \matlab+true+ it selects all diagonals from the first input \\
\matl{Yd} & 1-- (2 / $^\ddagger$) & 1 & \matlab+blkdiag+ \\
\matl{Zd} & 1--2 (2 / 1) & 1--3 (1) & \matlab+gcd+, element-wise with singleton expansion. With $1$ input and $1$ output, computes the greatest common divisor of all elements of the input \\
\matl{e} & 1-- (2 / 1) & 1 & (i) With $1$ input: \matlab+squeeze+. (ii) With more than $1$ input: \matlab+reshape+. If second input is logical, inputs beyond the second are ignored. Contiguous equal values in the second input indicate dimensions that will be collapsed; and a final \matlab+[]+ is implicit. If there are $2$ inputs and the second is non-logical and empty, it is replaced by the rounded-up square root of the number of elements of the first input, repeated; thus a square matrix is returned. If there are $2$ inputs and the second is a non-logical scalar, a final \matlab+[]+ is implicit. If size specification doesn't contain \matlab+[]+ (explicit or implicit), the first input is padded or truncated if needed; padding is done with \matlab+0+, \matlab+false+, \matlab+char(0)+ or \matlab+{[]}+ according to type of first input. If size specification contains \matlab+[]+ (explicit or implicit), the first input is padded if needed \\
\matl{Ye} & 1--2 (1 / 2) & 1 & (i) With one numerical or symbolic input: \matlab+expm+. (ii) With one logical input \matlab+M+: the input is interpreted as the adjacency matrix of a graph, which is made square if needed. The \matlab+logical+ version of the adjacency matrix of the power of the graph is computed with increasing exponent until the result no longer changes. This is the same as \matlab+logical(M*expm(M))+ except that numerical precision issues are avoided. (iii) With a logical input \matlab+M+ and a non-negative integer input \matlab+n+: \matlab+M+ interpreted as the adjacency matrix of a graph, which is made square if needed. The \matlab+logical+ version of the adjacency matrix of the power of the graph is computed with exponent \matlab+n+. \sa \matl{Y\textasciicircum{}} \\
\matl{Ze} & 1--2 (1 / 2) & 1 & (i) For $1$ input: \matlab+exp+. (ii) For $2$ inputs: Levenshtein distance between char, logical or numeric inputs. Either input can be a cell array, and then the result is a vector or matrix of distances. \\
\matl{f} & 1--3 (1) & 1--3 (1 / 2) & \matlab+find+ \\
\matl{Xf} & 1--4 (2 / 1) & 1 & (i) If 2 or more inputs: \matlab+strfind+. Works also when the first input is a numeric array or a cell array of numeric arrays. In this case each numeric array is linearized, and results are row vectors. (ii) If 1 input: output is a cell array of all non-empty substrings. Works also for numeric or cell arrays \\
\matl{Yf} & 1 & 1 & \matlab+factor+. For input $1$ it returns an empty array, not $1$ like \matlab+factor+ does. For negative input, \matlab+factor+ is applied to the absolute value of the input (so input $-1$ produces $1$). \sa \matl{YF} \\
\matl{g} & 1 & 1 & (i) For non-cell input: \matlab+logical+. Works also for complex numbers, using their absolute value. (ii) For cell input: \matlab+cell2mat+ \\
\matl{Xg} & 1-- (2) & 1-- ($^\square$) & \matlab+ndgrid+ \\
\matl{Yg} & 1--3 (1) & 1 & \matlab+gamma+ / \matlab+gammainc+ / \matlab+betainc+, depending on number of inputs \\
\matl{Zg} & 1--2 (1 / 2) & 1 & \matlab+gammaln+ / \matlab+betaln+, depending on number of inputs \\
\matl{h} & 0-- (2 / $^\ddagger$) & 1 & \matlab+horzcat+. If inputs are matrices with non-matching sizes they are linearized \\
\matl{Xh} & 0-- ($^\ddagger$ / 2) & 1 & concatenate into cell array (\matlab+{..., ...}+) \\
\matl{Yh} & 1--2 (2 / 1) & 1 & \matlab+hankel+. If any input is of type char: returns char output \\
\matl{Zh} & 3 & 1 & \matlab+hypergeom+. If a numerator coefficient is a nonpositive integer the series is truncated at the first index for which a zero appears in the numerator, regardless of the denominator \\
\matl{i} & 0--2 (0) & 1 & \matlab+input+ with content checking. If $0$ inputs: uses default prompt string. \sa \matl{j}. \\
\matl{Xi} & 1--5 (1) & 1--2 (1) & \matlab+urlread+. If the input string doesn't contain \matl{'://'}, the string \matl{'http://'} is prepended \\
\matl{Yi} & 1-- (1) & 1--3 (1 / 2) & \matlab+imread+ \\
\matl{j} & 0--1 (0) & 1 & \matlab+input(..., 's')+. If $0$ inputs: uses default prompt string. \sa \matl{i} \\
\matl{Xj} & 1 & 1 & \matlab+real+ \\
\matl{Yj} & 1 & 1 & \matlab+imag+ \\
\matl{Zj} & 1 & 1--2 (1 / 2) & (i) With one output: complex conjugate (\matlab+conj+). (ii) With two outputs: real and imaginary parts (\matlab+real+, \matlab+imag+) \\
\matl{k} & 1 & 1 & \matlab+lower+ for strings or cell arrays; \matlab+floor+ for numerical or symbolic arrays \\
\matl{Xk} & 1 & 1 & \matlab+upper+ for strings or cell arrays of strings; \matlab+ceil+ for numerical or symbolic arrays \\
\matl{Yk} & 2 & 1--3 (1 / 3rd) & closest value in second input to each value in first input, in terms of absolute difference. For equidistant values it picks the one that appears first (in linear order) in the second input. Second output gives resulting absolute differences. Third output gives linear indices of selected elements \\
\matl{l} & 0-- (0 / 2) & 1 & \matlab+ones+ (if $0$ inputs: produces output $1$) \\
\matl{Xl} & 2--3 (2 / 3) & 1 & (i) With $2$ inputs, the minimum of the two inputs is computed, element-wise with singleton expansion. (ii) With $3$ inputs, the function computes the minimum between the first and the second, and then the maximum between that and the third, element-wise with singleton expansion. (i, ii) Output is \matlab+char+ if first input is. \sa \matl{X<}, \matl{X>} \\
\matl{Yl} & 1--2 (1 / 2) & 1 & \matlab+log+. If $2$ inputs: second input specifies logarithm base \\
\matl{Zl} & 1 & 1--2 (1) & \matlab+log2+ \\
\matl{m} & 2--4 (2) & 1--2 (1 / 2nd) & \matlab+ismember+. It also works for cell arrays with numeric content, or with mixed char/numeric content. \sa \matl{Xm} \\
\matl{Xm} & 2--3 (2) & 1--2 (1 / 2nd) & \matlab+ismember(..., 'rows', ...)+. \sa \matl{m} \\
\matl{Ym} & 1--4 (1 / 2) & 1 & \matlab+mean+. This function allows flag strings in third and subsequent inputs to be replaced by numbers, as follows: 1: \matlab+'omitnan'+, 2: \matlab+'includenan'+, 3: \matlab+'double'+, 4: \matlab+'native'+, 5: \matlab+'default'+ \\
\matl{Zm} & 1--2 (2 / 1) & 1 & \matlab+lcm+, element-wise with singleton expansion. With $1$ input, computes the least common multiple of all elements of the input \\
\matl{n} & 1 & 1-- (1 / $^\times$) & \matlab+numel+. For several outputs, or a single output other than the first, gives \matlab+size+ along those dimensions \\
\matl{Xn} & 1--2 (2 / 1) & 1 & (i) If $2$ inputs: \matlab+nchoosek+. This interprets first input as number(s). If the inputs are arrays, the function is computed element-wise with singleton expansion. For values \matlab+n+ and \matlab+k+ in first and second inputs, if \matlab+n+ is less than \matlab+k+ the result is \matlab+0+. (ii) If $1$ input: multinomial coefficient. The input contains the lower terms; their sum is the upper term. \sa \matl{XN} \\
\matl{Yn} & 1--5 (2 / 4) & 1 & (i) If $1$ input: \matlab+poly+. (ii) If $2$ or more inputs: \matlab+interp1+. With $2$ or $3$ numeric inputs, flags \matlab+'linear'+ and \matlab+'extrap'+ are assumed. \matlab+'pp'+ option not supported. A finite numeric value in the fourth input is replaced by a flag string, as follows: 1: \matlab+'linear'+, 2: \matlab+'nearest'+, 3: \matlab+'next'+, 4: \matlab+'previous'+, 5: \matlab+'spline'+, 6: \matlab+'pchip'+, 7: \matlab+'cubic'+, 8: \matlab+'v5cubic'+ \\
\matl{o} & 1--3 (1 / 2) & 1 & (i) With one non-cell array, non-\matlab+double+ input: \matlab+double+. (ii) For cell array input behaves similarly to \matlab+char+: linearizes cell array, converts cell contents to \matlab+double+ and concatenates vertically, padding if needed. By default, padding is with zeros on the right. Optional second input indicates fill side: left if it evaluates to \matlab+false+, right if it evaluates to \matlab+true+. Third optional input specifies fill value. (iii) With one input of type \matlab+double+: \matlab+mod(..., 2)+ \\
\matl{Xo} & 1 & 1 & \matlab+int64+ \\
\matl{Yo} & 1--3 (1 / 2) & 1 & (i) For numeric arrays: \matlab+round+. Third input can be a number instead of a string, as follows: 1: \matlab+'significant'+, 2: \matlab+'decimals'+. (ii) For character arrays or cell arrays of strings: change case. Optional second input specifies in which characters or cells the case will be changed \\
\matl{Zo} & 1 & 1 & \matlab+fix+ \\
\matl{p} & 1--3 (1 / 2) & 1 & \matlab+prod+. If first input is \matlab+char+ it is converted to \matlab+double+. \sa \matl{Xp} \\
\matl{Xp} & 1--3 (1) & 1 & \matlab+prod(..., 1, ...)+. If first input is \matlab+char+ it is converted to \matlab+double+. \sa \matl{p}. \\
\matl{Yp} & 1--4 (1 / 2) & 1 & \matlab+cumprod+. Allows first input to be char. If $3$ numeric inputs: the third input is interpreted as a minimum which is applied after the product at each position. If $4$ numeric inputs: last two inputs are minimum and maximum respectively \\
\matl{Zp} & 1 & 1 & For input with non-negative entries: \matlab+isprime+. For input with non-positive entries: Euler's totient function for absolute value of each entry \\
\matl{q} & 1 & 1 & \matlab+(...)-1+ \\
\matl{Xq} & 2--3 (2 / 3) & 1 & \matlab+quantile+ \\
\matl{Yq} & 1 & 1 & For input with positive entries: $n$-th prime for each value $n$ in the input array. For input with non-positive entries: next prime for absolute value of each entry. \sa \matl{Zq} \\
\matl{Zq} & 1 & 1 & \matlab+primes+. \sa \matl{Yq} \\
\matl{r} & 0-- (0 / 2) & 0--1 (1) & \matlab+rand+ \\
\matl{Xr} & 0-- (0 / 2) & 0--1 (1) & \matlab+randn+ \\
\matl{Yr} & 1-- (1 / 2) & 1 & \matlab+randi+. If the first input is a vector of $2$ elements it is automatically sorted. With $2$ inputs, if the second input is a scalar \matlab+n+ it is transformed into \matlab+[1 n]+ \\
\matl{Zr} & 1--4 (2 / 3) & 1 & (i) With $2$ or more inputs: \matlab+randsample+. Non-vector first input is linearized to a column. Does not support stream specification. (ii) With $1$ input: shuffle the input array, that is, apply a random permutation to it. This is done linearly, that is, regardless of dimensions. The shape of the array is preserved.  \\
\matl{s} & 1--4 (1 / 2) & 1 & \matlab+sum+. This function allows flag strings in third and subsequent inputs to be replaced by numbers, as follows: 1: \matlab+'omitnan'+, 2: \matlab+'includenan'+, 3: \matlab+'double'+, 4: \matlab+'native'+, 5: \matlab+'default'+. \sa \matl{Xs} \\
\matl{Xs} & 1--3 (1) & 1 & \matlab+sum(..., 1, ...)+. This function allows flag strings in third and subsequent inputs to be replaced by numbers, as follows: 1: \matlab+'omitnan'+, 2: \matlab+'includenan'+, 3: \matlab+'double'+, 4: \matlab+'native'+, 5: \matlab+'default'+. \sa \matl{s} \\
\matl{Ys} & 1--4 (1 / 2) & 1 & \matlab+cumsum+. Allows first input to be char. Optional third and fourth numerical scalar inputs are interpreted as minimum and maximum values, which are applied after the sum at each position. These can be substituted by a 2-element numerical array as second or third input \\
\matl{Zs} & 1--4 (1 / 2) & 1 & (i) \matlab+std+. (ii) With $2$ inputs \matlab+x+, \matlab+k+ and \matlab+k+ equal to $2$ or $3$, gives \matlab+cov(x, k-2)+. (iii) With $2$ inputs \matlab+x+, \matlab+k+ and \matlab+k+ equal to $4$ or $5$, gives \matlab+skewness(x(:), k-4)+. (iv) With $2$ inputs \matlab+x+, \matlab+k+ and \matlab+k+ equal to $6$ or $7$, gives \matlab+kurtosis(x(:), k-6)+ \\
\matl{t} & 0-- (1) & 0 & duplicate elements in stack. The duplicated elements are those specified as inputs. \sa \matl{y} \\
\matl{Zt} & 3 & 1 & (i) If first input is char or a cell array: \matlab+strrep+. Numerical second and third inputs are converted to char. If the first input is char it is linearized into a row vector. (ii) The first input can be a numerical or logical array (not a cell array of numeric or logical arrays). In this case the other inputs can be char, numeric or logical (not cell); each input array is linearized; result is a row vector; and result has the type of the third input, even if no substitutions have been actually done \\
\matl{u} & 1--4 (1) & 1--4 (1) & \matlab+unique+, with \matlab+'stable'+ option by default. It includes a fourth output with the count for each unique element. It also allows cell array input with numeric content, or mixed char/numeric content. In this case it only works with a single output, either in sorted or stable mode. This function allows flag strings in second and subsequent inputs to be replaced by numbers, as follows: 1: \matlab+'stable'+, 2: \matlab+'sorted'+, 3: \matlab+'last'+, 4: \matlab+'first'+, 5: \matlab+'rows'+. \sa \matl{Xu} \\
\matl{Xu} & 1--3 (1) & 1--4 (1) & \matlab+unique(..., 'rows', ...)+, with \matlab+'stable'+ option by default. It includes a fourth output with the count for each unique row. This function allows flag strings in second and subsequent inputs to be replaced by numbers, as follows: 1: \matlab+'stable'+, 2: \matlab+'sorted'+, 3: \matlab+'last'+, 4: \matlab+'first'+. \sa \matl{u} \\
\matl{Zu} & 2 & 1 & \matlab+strjust+. Strings in second input can be replaced by numbers, as follows: 1: \matlab+'right'+, 2: \matlab+'left'+, 3: \matlab+'center'+ \\
\matl{v} & 0-- ($^\ddagger$ / 2) & 1 & \matlab+vertcat+. If inputs are matrices with non-matching sizes they are linearized \\
\matl{Yv} & 1--2 (1) & 1--3 (1 / 2nd) & (i) If input is numeric or symbolic: with $1$ or $2$ outputs: \matlab+eig+. With $3$ outputs: \matlab+svd+. If output specification is \matl{FTF} (keep second output only), \matlab+svd+ is called with a single output, which gives a vector with the singular values. (ii) If input is a string or cell array of strings: \matlab+strtrim+. For string or char array input, char $0$ also counts as whitespace \\
\matl{Zv} & 1--2 (1 / 2) & 1 & (i) For one scalar numeric or symbolic input \matlab+n+: symmetric range. For nonnegative \matlab+n+ it rounds down and then gives \matlab+[1:n n-1:-1:1]+; for negative \matlab+n+ it removes sign, rounds down and gives \matlab+[n:-1:1 2:n]+. (ii) For two numeric inputs: applies symmetric range indexing to the first input along each dimension specified by the (linearized) second input. The indexing along a given dimension doesn't repeat the last entry if the number specifying the dimension is positive, and repeats it if negative. (iii) For one string or cell array input: \matlab+deblank+ \\
\matl{w} & 0-- (2) & 0 & swap elements in stack. \sa \matl{b} \\
\matl{x} & 0-- (1) & 0 & delete elements from stack. The deleted elements are those specified as inputs  \\
\matl{Xx} & 0--1 (0 / 1) & 0 & \matlab+clc+. With $1$ input \matlab+x+: \matlab+pause(x/10)+ and then \matlab+clc+. Inputs \matl{F} and \matl{T} are interpreted as \matl{2.5} and \matl{10} respectively \\
\matl{Zx} & 0 & 0 & resets stack, input/output specifications and clipboards to their initial values \\
\matl{y} & 0-- (2 / 3) & 0 & duplicate one element in stack. The duplicated element is the lowest among those specified as inputs, if any. \sa \matl{t} \\
\matl{Xy} & 1--4 (1) & 1 & \matlab+eye+ (matrix with ones on diagonal and zeros elsewhere) \\
\matl{Yy} & 2 & 1 & \matlab+hypot+, element-wise with singleton expansion \\
\matl{Zy} & 1--2 (1) & 1-- (1) & \matlab+size+ \\
\matl{z} & 1 & 1 & (i) For non-cell input: \matlab+nnz+: number of nonzero elements. (ii) For cell array input: \matlab+cellfun(@nnz, ...)+: number of nonzero elements of each cell \\
\matl{Xz} & 1 & 1 & \matlab+nonzeros+. If input is a string or char array: removes whitespace, including \matlab+char(0)+; the result is a string. If input is cell array of strings: operates on each string \\
\matl{X\{} & 1--2 (1 / 2) & 1 & \matlab+num2cell+ \\
\matl{Y\{} & 2-- (3 / 2) & 1 & \matlab+mat2cell+ \\
\matl{Z\{} & 1 & 1 & \matlab+cellstr+ for \matlab+char+ input, or \matlab+mat2cell(x, ones(size(x,1),1), size(x,2),...,size(x,ndims(x)))+ for non-\matlab+char+ arrays of any number of dimensions \\
\matl{|} & 1--2 (1 / 2) & 1 & If $1$ input: \matlab+abs+. If $2$ inputs, second input non-zero: \matlab+norm+. If $2$ inputs, second input zero: \matlab+det+, or \matlab+sqrt(det(A*A.'))+ for first input \matlab+A+ non-square \\
\matl{X|} & 2--4 (2 / 3) & 1--3 (1) & \matlab+union+. Uses the \matlab+'stable'+ flag by default. If one input is char and the other is numeric, the latter is converted to char. This function allows flag strings in third and subsequent inputs to be replaced by numbers, as follows: 1: \matlab+'rows'+, 2: \matlab+'stable'+, 3: \matlab+'sorted'+ \\
\matl{Y|} & 0-- (2 / $^\ddagger$) & 1 & \matlab+|+ (\matlab+or+), element-wise with singleton expansion \\
\matl{Z|} & 2--3 (2) & 1 & \matlab+bitor+ (bitwise 'or'), element-wise with singleton expansion. If first or second inputs are \matlab+char+ they are converted to \matlab+double+. \matlab+double+ inputs are rounded. If the inputs are \matlab+double+ they can have negative values, and in that case they must be in the range from \matlab+-2^52+ up to \matlab+2^52-1+. \sa \matl{Z\&}, \matl{Z\textasciitilde{}} \\
\matl{Z\}} & 1--2 (1 / 2) & 0-- ($^\bigtriangledown$) & split array into subarrays along the first non-singleton dimension. With $2$ inputs: split into subarrays along the dimension indicated by the second input \\
\matl{\textasciitilde{}} & 1 & 1 & \matlab+~+ (\matlab+not+) \\
\matl{X\textasciitilde{}} & 2--4 (2 / 3) & 1--3 (1) & \matlab+setxor+. Uses the \matlab+'stable'+ flag by default. If one input is char and the other is numeric, the latter is converted to char. This function allows flag strings in third and subsequent inputs to be replaced by numbers, as follows: 1: \matlab+'rows'+, 2: \matlab+'stable'+, 3: \matlab+'sorted'+. \sa \matl{X-} \\
\matl{Y\textasciitilde{}} & 2 & 1 & \matlab+xor+, element-wise with singleton expansion \\
\matl{Z\textasciitilde{}} & 2--3 (2) & 1 & \matlab+bitxor+ (bitwise 'xor'), element-wise with singleton expansion. If first or second inputs are \matlab+char+ they are converted to \matlab+double+. \matlab+double+ inputs are rounded. If the inputs are \matlab+double+ they can have negative values, and in that case they must be in the range from \matlab+-2^52+ up to \matlab+2^52-1+. \sa \matl{Z\&}, \matl{Z|} \\
